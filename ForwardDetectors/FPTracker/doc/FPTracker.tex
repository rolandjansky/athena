\documentclass[a4paper,12pt]{article}
\usepackage{epsfig}
\usepackage{epsf}
\usepackage{graphicx}
\usepackage{draftcopy}
\oddsidemargin 0cm
\evensidemargin 0cm
\textwidth 16.0cm
\parindent 0pt
\parskip 1ex
\newtheorem{usecase}{Use Case}
\newcommand{\tbd}{\textbf{To Be Discussed:\_}}
\newcommand{\eg}{{\it e.g.}}
\newcommand{\ie}{{\it i.e.}}

\begin{document}
\bibliographystyle{plain}
\title{FPTracker: An OO version of FPTrack}
\author{Peter Bussey, Peter Sherwood}
\date{14/08/09 v0.2}
\maketitle

\section{Introduction}
The C++ version of Peter Bussey's original FPTrack program reads LHC
configuration data files to fill a number of arrays containing information
on the location, type and strength of the beam line magnets and collimator settings.

It has an event loop in which particles are sent down the beam line. The magnet
information is obtained from the arrays. Code to calculate quantities for
various physics studies is included in the event loop.

FPTracker is a recast of FPTrack as an object-oriented program. The
associations among the data of the various arrays, which is implicit in
the array index manipulations in FPTrack, are made explicit in FPTracker
as the data for a particular item, such as a magnet, is collected together
in the object representing the item.

FPTracker does not contain analysis code. Instead it acts as a function which
takes a particle and transports it to an arbitrary location down the beam line.

The demonstration program, FPTrackerMain.cxx, instantiates two Beamline objects
(one each for $z>0$, $z<0$) which contain stop planes - planes at
which the tracking calculation is terminated. FPTracker particles are generated in
a calling program, and are passed to beam-line.track(particle) method. The particle
may then be interrogated to find if it was lost, or whether it struck a stop plane.
The final coordinates of the particle are available for further analysis by the
the calling program.

\section{FPTracker classes}
FPTracker defines a small number of interfaces, and the necessary concrete
class implementations. A beam line is a container of IBeamElement objects.
Concrete implementations of IBeamElement include magnets, collimators and
planes. The IBeamElement interface contains methods to allow the tracking
of the particle down the beam line.

The function magnetSet() reads in the magnet data in the same manner as
FPTrack. The function magnetFactory instantiates the various component 
objects (Benders for dipoles, Benders and Focusers for quadrupole magnets) and
returns assembled magnets.

The beam line elements are added to a container (the beam line), and are ordered
by their z-coordinate.

The classes for FPTracker are shown in fig~\ref{fig:FPTrackerClasses}.
\begin{figure}
  \begin{center}
    \epsfig{file=./FPTracker.eps,width=0.9\textwidth}{
      \caption{\label{fig:FPTrackerClasses} 
        Classes for FPTracker.
      }
    }
  \end{center} 
\end{figure}

\section{Object Lifetime}
The objects representing beam elements (magnets, collimators and planes) as well as their
internal objects, have a the same lifetime as the Beamline object. Technically, this is
done by using Boost shared (const) pointers, and these classes have no bare pointers. The copy constructor
and copy assignment operators are the default methods provided by the compiler.
Thus, for example, if a beamline object is copied, a new beamline object is produced. However,
a Beamline is implemented as a sequential container of \verb+shared_ptr<IBeamElement>+ no new 
beam elements are constructed - only the reference count of the pointer is incremented.
As the pointers are const, no changes can be made to the objects components.    

\section{Transport}
A particle produced by a calling routine (\eg\ an analysis program),
and is passed to the beam line track method. The particle is tracked down
the beam line by iterating over the beam line elements until either the 
particle goes out of aperture, or the particle traverses a plane which marks the
end of the tracking. The particle is available to the calling routine for
interrogation: did the the particle leave the beam pipe? if so where? if not
what are its position, direction and momentum at the termination plane?

The FPTrackerMain code is:
\begin{verbatim}
#include "setupBeamline.h"
#include "Beamline.h"
#include "Particle.h"
#include <iostream>
#include <fstream>

int main(){

  FPTracker::Beamline beamline_0 = FPTracker::setupBeamline(0);
  FPTracker::Beamline beamline_1 = FPTracker::setupBeamline(1);

  std::cout<<"\n\n-- side 0 --";
  std::cout<<beamline_0<<std::endl;
  std::cout<<"\n\n-- side 1 --";
  std::cout<<beamline_1<<std::endl;

  FPTracker::Particle p01(0., 0., 0., 0., 0., -7000.);
  std::cout<<"fpTracker: "<<p01<<std::endl;
  p01.direction().z()<0. ? beamline_0.track(p01):beamline_1.track(p01);
  std::cout<<"fpTracker: "<<p01<<std::endl;
}
\end{verbatim}

The final line of output, which gives the result of tracking the particle is:
  
\begin{verbatim}
fpTracker: Part: pos: x -0.097008 y  0.002286 z -420.00 dir  
                      xp -0.000000 yp  0.000032 zp -1.000000 mom: 7000 lost false
\end{verbatim}

The actions of FPTracker are summarised in fig~\ref{fig:ActivityOverview}.

\begin{figure}
  \begin{center}
    \epsfig{file=./ActivityOverview.eps,width=0.9\textwidth}{
      \caption{\label{fig:ActivityOverview} 
        Overview of the operation of FPTracker.
      }
    }
  \end{center} 
\end{figure}

\subsection{Magnet Tracking}
Of all the elements in the beam-line, magnets have the most complex tracking algorithm. The particle is first
transported to the front face (the face closest to the interaction point) by projecting from particle from its
current position (generally the back-end of the previous element) in a straight line. The particle is checked for being
in the aperture. The aperture checking is done using a position measured with respect to the magnet position. If the
particle is in aperture, the magnet's bender object is invoked to bend the particle trajectory. In the case of
quadrupole magnets, a further object, a focus-er, is used to calculate the bend. Bending and focusing is done in 
coordinates measured with respect to the magnet position. To facilitate the coding, the Particle class has methods
that convert between the magnet and overall coordinate systems. 

The appropriate bender and (for quadrupole benders) the appropriate vertical or horizontal focuser is placed in the
magnet by the magnetFactory function.

The sequence of calls is shown in fig~\ref{fig:MagnetTrackSeq}.

\begin{figure}
  \begin{center}
    \epsfig{file=./MagnetTrackSeq.eps,width=0.9\textwidth}{
      \caption{\label{fig:MagnetTrackSeq} 
        Sequence diagram showing particle tracking through a magnet.
      }
    }
  \end{center} 
\end{figure}
\end{document}