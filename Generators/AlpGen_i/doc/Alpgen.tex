\documentclass[11pt]{article}
\newdimen\SaveHeight \SaveHeight=\textheight
\textwidth=6.5in
\textheight=8.9in
\textwidth=6.5in
\textheight=9.0in
\hoffset=-.5in
\voffset=-1in
\def\topfraction{1.}
\def\textfraction{0.}   
\def\topfraction{1.}
\def\textfraction{0.}           

\def\jimmy{{\sc Jimmy}}
\def\herwig{{HERWIG}}
\def\pythia{{\sc Pythia}}
\def\alpgen{{AlpGen}}


\begin{document}
\title{\alpgen\_i: An interface between \alpgen\ and Athena\\
Version in release 10.4.0 and later}
\author{  Ian Hinchliffe (I\_Hinchliffe@lbl.gov), Georgios Stavropoulos (George.Stavropoulos@cern.ch) and Jon Butterworth (J.Butterworth@ucl.ac.uk)}
%\today

\maketitle           

\section{Introduction}

\noindent
This package runs \alpgen\  from within Athena. See the examples in 
\\ {\bf \alpgen\_i/share/jobOptions.AlpgenPythia.py } 
and 
\\ {\bf \alpgen\_i/share/jobOptions.AlpgenHerwig.py }  
which show how to read Alpgen events, add parton showers and
hadronize them using \pythia\ or \herwig. Note that on ATLAS we use the
``new'' (i.e. \pythia 6.4) shower and this does not work with the \alpgen\
parton shower matching, so \herwig\ should be used in general.

\bigskip

\noindent
Users must first run \alpgen\ in standalone mode and make a file of
events. An athena job then takes these events hadronizes them and
passes them down the Athena event chain. The events must be made with
a version of \alpgen\ that is compatible, recent versions that support
the Les Houches interface should be acceptable. The current release is
compatible with versions 2.13 to 2.11. It is recommended that you use
the latest compatible version of the standalone \alpgen.

Within the {\tt GeneratorModules}, {\tt m\_ExternalProcess = 4} is used to
signify \alpgen\ to both the \herwig\ and \pythia\ interfaces.

\bigskip

\section{\alpgen\ and \pythia}

\noindent
To hadronize {\bf \alpgen} generated events with \pythia, you need to link the file with the input parameters 
and the one with the unweighted events produced by \alpgen\ to the files inparmAlpGen.dat and alpgen.unw\_events 
respectively. Then you only need to run athena with the jobOptions file jobOptions.AlpgenPythia.py by typing 
in the prompt \\ {\it athena jobOptions.AlpgenPythia.py}


\bigskip

\section{\alpgen\ and \herwig}

\noindent
To hadronize {\bf \alpgen} generated events with \herwig, you need to
link the file with the input parameters and the one with the
unweighted events produced by \alpgen\ to the files inparmAlpGen.dat
and alpgen.unw\_events respectively. Then you only need to run athena
with the jobOptions file jobOptions.AlpgenHerwig.py by typing in the
prompt \\ {\it athena jobOptions.AlpgenHerwig.py}\\

\medskip

\noindent
The program flow in \herwig\ is a little complex, especially at the
end of a run. The interface Herwig.cxx (in Herwig\_i) calls HWHGUP (a
\herwig\ routine) which calls UPEVNT (modified by ATLAS to make sure
it calls the right routine in usealpgen.f, which is in
AlpGen\_i). When the end of a file is found, UPEVNT calls ALSFIN (in
atoher\_65.f in Herwig\_i), which then calls HWUGUP (\herwig\
routine). ATLAS uses a modified version of HWUGUP which instead of
abruptly stopping the program (which is bad for Athena!) sets IERROR
and GENEV such that events generation is terminated more
gracefully. This still leads to some confusing apparent error messages
from Athena, but the \herwig\ and \alpgen\ output is at least
consistent.

\medskip

\noindent
To specify \alpgen\ running with \herwig\ you need to set the line\\
{\tt job.Herwig.HerwigCommand = ["iproc alpgen"]} \\
in your jobOptions file.  It is also possible to specify Higgs decay information 
for the events by adding a second argument to the {\tt iproc} parameter, 
which will be interpreted as the ID value (last two digits of IPROC) controlling 
the Higgs decay.  e.g. \\
{\tt job.Herwig.HerwigCommand = ["iproc alpgen 12"]}\\ 
would force any Higgs particles present in your input events to decay to two photons. See the \herwig\ manual for more information.


\end{document}







