\documentclass[11pt]{article}
\newdimen\SaveHeight \SaveHeight=\textheight
\textwidth=6.5in
\textheight=8.9in
\textwidth=6.5in
\textheight=9.0in
\hoffset=-.5in
\voffset=-1in
\def\topfraction{1.}
\def\textfraction{0.}   
\def\topfraction{1.}
\def\textfraction{0.}           
\begin{document}
\title{\bf Ampt\_i: An Interface Between AMPT Heavy Ion Event Generator and Athena}
\author{ Alexandre (Sasha) Lebedev (lebedev@iastate.edu)}
%\today

\maketitle           

\section{Introduction}

This package runs Ampt event generator in Athena frameworkand puts the events into the
transient store in HepMC format. Optional HepMC ascii output and original text outputs
are also available.

AMPT (A Multi-Phase Transport model) uses HIJING for generating the initial conditions, 
Zhang's Parton Cascade (ZPC) for modeling partonic scatterings, the Lund string fragmentation 
model or a quark coalescence model for hadronization, and A Relativistic Transport (ART) model 
for treating hadronic scatterings. 

The model is intended to give a coherent and realistic description of the dynamics of relativistic 
heavy ion collisions. In particular, the model describes well pseudorapidity and transverse momentum 
distributions of charged hadrons as a function of collision centrality at LHC energies. 
Detailed description of the model was published in Z.W. Lin et al., Phys. Rev. C72, 064901 (2005). 
The original AMPTcode is written in FORTRAN and uses text files as input and output. 

\section {Running Ampt}

The Ampt event generator is run from jobOptions script, see for example, 

{\em Generators/Ampt\_i/share/jobOptions\_Ampt.py } as follows:

\begin{verbatim}
athena  jobOptions_Ampt.py >& logfile.txt &
\end{verbatim}



\section {Input Parameters}

\subsection{Event Parameters}

\begin{tabular}{lll}
Parameter  & Meaning  & Default value \\
%\hline
{\bf EFRM}  &  $\sqrt{s}$   &    2560. GeV \\
{\bf Frame} & collision frame  &     CMS \\
{\bf Proj}   & type of projectile  &   A \\
{\bf Targ} & type of target          &    A \\
{\bf ATarg}  & target A number     &           197  \\
{\bf ZTarg}  & target Z number       &   79  \\
{\bf AProj} & projectile A number     &           197 \\
{\bf ZProj}  & projectile Z number     &     79   \\
{\bf Bmin}  & minimum inpact parameter  &   0. \\
{\bf Bmax}  & maximum inpact parameter   &  14.6 fm (Min. Bias Au+Au) \\
\end{tabular}

\subsection{EventVertex Randomization}

\begin{tabular}{lll}
Parameter & Meaning & Default value \\
{\bf randomizeVertices} &  flag to set up event vertex randomization   &    False  \\
{\bf selectVertex} & if true, fixed vertex is used. Vertex position & False \\
                          & is defined by Xvtx, Yvtx and Zvtx input parameters &   \\
{\bf wide} & allow randoms off the beamline (out of the pipe &  False \\
{\bf Xvtx} & if {\bf selectVertex} is True, this is vertex X position  & 0. \\
{\bf Yvtx} & if {\bf selectVertex} is True, this is vertex Y position  &  0. \\
{\bf Zvtx} & if {\bf selectVertex} is True, this is vertex Z position  &  0.  \\
\end{tabular}

\subsection{Miscellaneous Parameters}

\begin{tabular}{lll}
Parameter & Meaning & Default value \\
{\bf stringSwitch} & Switch between standard AMPT (1) and string melting version (4)  &    4 \\
{\bf writeHepMC}  & Switch to write out ASCII output in HepMC format            &       False \\
{\bf writeAmpt}  & Switch to write out original AMPT text output files       &          False \\
{\bf decayKs}  & Decay or not K-short    &    0 (do not decay) \\
{\bf decayPhi}  & Decay or not $\phi$-meson   &     1 (yes) \\
{\bf decayPi0}  & Decay or not $\pi^0$    &    0 (do not decay) \\
{\bf iShadow}   & Flag to enable to modify nuclear shadowing (1=yes, 0=no)  &  0  \\
{\bf dShadow}  & Factor used to modify nuclear shadowing   &  1.0 \\
{\bf iPhiRP}   & Flag for random orientation of reaction plane (0=no, 1=yes) &  0 \\
\end{tabular}

\bigskip

To modify, for example, collision energy ($\sqrt{s}$), use the following lines in the jobOptions sctipt:

{\em topAlg.Ampt.EFRM = 1000.; }



\section {Random Numbers}

Ampt\_i is using the AtRndmGenSvc Athena Service to provide the necessary random numbers.
Ampt\_i is using two streams: AMPT\_INIT and AMPT. The first stream is used to provide
 random numbers for the initialization phase of Hijing and the second one for the event generation. The user
 can set the initial seeds of each stream via the following option in the jobOption file.

 \begin{verbatim} 
 AtRndmGenSvc.Seeds = [``AMPT_INIT 2345533 9922199'', ``AMPT 5498921 659091'']
 \end{verbatim}

 The above sets the seeds of the AMPT\_INIT stream to 2345533 and 9922199 and of the AMPT one to
 5498921 and 659091. If the user will not set the seeds of a stream then the AtRndmGenSvc will use default
 values.

 The seeds of the Random number service are saved for each event in the HepMC Event record and they are printed
 on screen by DumpMC. In this way an event can be reproduced easily. The user has to rerun the job by simply seting
 the seeds of the HIJING stream (the seeds of the AMPT\_INIT stream should stay the same) to the seeds of that
 event.

 




\end{document}






