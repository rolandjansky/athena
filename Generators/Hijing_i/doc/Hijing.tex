\documentclass[11pt]{article}
\newdimen\SaveHeight \SaveHeight=\textheight
\textwidth=6.5in
\textheight=8.9in
\textwidth=6.5in
\textheight=9.0in
\hoffset=-.5in
\voffset=-1in
\def\topfraction{1.}
\def\textfraction{0.}   
\def\topfraction{1.}
\def\textfraction{0.}           
\begin{document}
\title{Hijing\_i: An interface between Hijing and Athena\\
Version in release 6.5.0 and later}
\author{ Georgios Stavropoulos (George.Stavropoulos@cern.ch)}
%\today

\maketitle           

This package runs Hijing from within Athena, puts the events into the
transient store in HepMC format. See the documentation on GenModule  for
general information. The note refers only to Hijing specific
material. The External/Hijing package is used to set up the paths to
the Hijing library. This works with Hijing version 1.381.

The module is activated from the jobOptions service. \\See the example
in {\bf Hijing\_i/share/jobOptions.hijing.py }

The hijing parameters are set from the job options service. The
default parameters initialize Hijing for pp colisions at c.m. energy of 14 TeV.

{ \bf Note that all parameters passed to Hijing are in the units specified
in the Hijing manual. In particular, energies and masses are in GeV,
not the standard atlas units of MeV. }

The default jobOptions.hijing.py file will have been copied to your TestRelease
area when you set up athena under CMT.
The following is needed if you wish to run Hijing
\begin{verbatim}
theApp.Dlls  += [ "Hijing\_i"]
theApp.TopAlg = ["Hijing"]
\end{verbatim}
The initialization parameters can be changed via the following line in the jobOptions.py
file.

\begin{verbatim}Hijing.Initialize = ["variable index value", "variable1 index1 value1"]
\end{verbatim} 

Each quoted string sets one parameter. You can have as many as you like
seperated by commas. 
{\bf variable} must be one of the following variable names and
must be in lower case.\\
efrm\\
frame\\
proj\\
targ\\
iap\\
izp\\
iat\\
izt\\
bmin\\
bmax\\
nseed\\
hipr1\\
ihpr2\\
hint1\\
ihnt2\\
An error message is returned if the specified variable is not in the above list. The job continues
to run but the behaviour may not be what you expect.\\
The variables efrm to izt are the input parameters to Hijing initialization routine hijset
(see http://www-nsdth.lbl.gov/$\sim$xnwang/hijing/). Their default values are:
efrm=14000., frame='CMS', proj='P', targ='P', iap=izp=iat=izt=1.\\
The variables bmin and bmax are the input parameters to hijing routine\\
(see http://www-nsdth.lbl.gov/$\sim$xnwang/hijing/)
and their default
value is bmin=bmax=0.\\
The variables hipr1 to ihnt2 are the arrays in the hiparnt common block. A detailed explanation
of these variables can be found in http://www-nsdth.lbl.gov/$\sim$xnwang/hijing/ \\

{\bf index} is the index in the arrays hipr1, ihpr2, hint1 and ihnt2. For the other variables
has no meaning and SHOULD be omitted.\\

{\bf value} is the parameter's value.\\

Example:\\
The following generates events for Au+Au collisions at 200 GeV c.m energy, switches off jet quenching,
switches on triggered jet production and sets the pt range of the triggered jets.
\begin{verbatim}Hijing.Initialize = ["efrm 200", "frame CMS", "proj A", "targ A", 
                           "iap 197", "izp 79", "iat 197", "izt 79",
                           "ihpr2 4 0", "ihpr2 3 1", "hipr1 10 -20"]
\end{verbatim} 


{\large \bf Random Numbers}\\

 Hijing\_i is using the AtRndmGenSvc Athena Service to provide to Hijing (via the ran function,
 found in Hijing\_i/src/ran.F) the necessary random numbers. This service is using the RanecuEngine of CLHEP,
 and is based on the ``stream'' logic, each stream being able to provide an idependent sequence of random
 numbers. Hijing\_i is using two streams: HIJING\_INIT and HIJING. The first stream is used to provide
 random numbers for the initialization phase of Hijing and the second one for the event generation. The user
 can set the initial seeds of each stream via the following option in the jobOption file.

 \begin{verbatim} 
 AtRndmGenSvc.Seeds = [``HIJING_INIT 2345533 9922199'', ``HIJING 5498921 659091'']
 \end{verbatim}

 The above sets the seeds of the HIJING\_INIT stream to 2345533 and 9922199 and of the HIJING one to
 5498921 and 659091. If the user will not set the seeds of a stream then the AtRndmGenSvc will use default
 values.

 The seeds of the Random number service are saved for each event in the HepMC Event record and they are printed
 on screen by DumpMC. In this way an event can be reproduced easily. The user has to rerun the job by simply seting
 the seeds of the HIJING stream (the seeds of the HIJING\_INIT stream should stay the same) to the seeds of that
 event.

 Additionaly the AtRndmGenSvc is dumping into a file (AtRndmGenSvc.out) the seeds of all the streams at the end
 of the job. This file can be read back by the service if the user set the option
 \begin{verbatim} AtRndmGenSvc.ReadFromFile = true \end{verbatim} (default = false). In this case the file
 AtRndmGenSvc.out is read and the streams saved in this file are created with seeds as in this file. The name of
 the file to be read can be set by the user via the option
 \begin{verbatim} AtRndmGenSvc.FileToRead = MyFileName \end{verbatim}

 The above file is also written out when a job crashes. {\bf This last option (when job crashing) is currently not
   working, waiting for a modification in Athena}.

 The {\bf Hijing\_i/share/jobOptions.hijing.py } contains the above options.\\

{\large \bf User modifications}\\

If you are trying to replace an existing routine that is in the Hijing library this is straightforward.
Assume that you are trying to replace test.f that exists in Hijing.
Check out Hijing\_i under CMT,  (use cmt co -r Hijing\_i-xx-xx-xx  Generators/Hijing\_i where
xx-xx-xx is the version in the release that you are running against), put your version of test.f
into the /src area of the checked out code. Then in the /cmt area edit the requirements file and add test.f
into the list of files that get complied. Note that each generator has its own library. You must therefore put
your file in the right place. \\


\end{document}







