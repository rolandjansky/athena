\documentclass[11pt]{article}
\newdimen\SaveHeight \SaveHeight=\textheight
\textwidth=6.5in
\textheight=8.9in
\textwidth=6.5in
\textheight=9.0in
\hoffset=-.5in
\voffset=-1in
\def\topfraction{1.}
\def\textfraction{0.}   
\def\topfraction{1.}
\def\textfraction{0.} 
          
\begin{document}
\title{Dpemc\_i: An interface between Dpemc and Athena}
\author{ Vojtech Juranek (Vojtech.Juranek@cern.ch)}
\date{March, 2006}

\maketitle       

This package runs Dpemc from within Athena, puts the events into the
transient store in HepMC format. 
As Dpemc is an extension to Heriwg, the user should read documentation of Herwig\_i interface. 

The module is activated from the jobOptions service. See the example in {\bf Generators/Herwig\_i/share/jobOptions.dpemc.py }.
The following is needed if you wish to run Dpemc
\begin{verbatim}
theApp.DLLs  += [ "Dpemc\_i" ]  
theApp.TopAlg = ["Dpemc"]
\end{verbatim}
The parameters are passed via the following line in the jobOptions.py
file.
\begin{verbatim}
Dpemc.DpemcCommand = ["variable index value"]
\end{verbatim} 
In addition to Herwig variables there are several variables, which can be changed.
The list of Dpemc variables that can be changed is as follows:
{\bf 
typepr,
typint,
gapspr,
prospr,
cdffac,
nstru
}.
For more details see http://boonekam.home.cern.ch/boonekam/dpemc.htm .

\end{document}
