\documentclass[11pt]{article}
\newdimen\SaveHeight \SaveHeight=\textheight
\textwidth=6.5in
\textheight=8.9in
\textwidth=6.5in
\textheight=9.0in
\hoffset=-.5in
\voffset=-1in
\def\topfraction{1.}
\def\textfraction{0.}   
\def\topfraction{1.}
\def\textfraction{0.}           
\begin{document}
\title{Herwig\_i: An interface between HERWIG/JIMMY and Athena}
\author{ 
Ian Hinchliffe (I\_Hinchliffe@lbl.gov), \\
Georgios Stavropoulos (George.Stavropoulos@cern.ch) \\
Jon Butterworth (J.Butterworth@ucl.ac.uk)}
%\today

\maketitle       
 
\noindent
This package runs HERWIG, with optional underlying event simulation
from JIMMY, from within Athena, puts the events into the transient
store in HepMC format. See the GeneratorModules documentation for
general information. The note refers only to HERWIG specific
material. The External/Herwig package is used to set up the paths to
the HERWIG library. The current HERWIG version is 6.510, and the
current JIMMY version is 4.31.

Note that all parameters passed to HERWIG are in the units specified
in the HERWIG manual. In particular, energies and masses are in GeV,
not the standard atlas units of MeV. 

The module is activated from the jobOptions service. See the example in 
\begin{verbatim}
Generators/Herwig\_i/share/jobOptions.herwig.py
\end{verbatim}
which leaves JIMMY turned off, and 
\begin{verbatim}
Generators/Herwig\_i/share/jobOptions.jimmy.py
\end{verbatim}
for an example to run with it on. Note that it is recommended to turn
JIMMY on and leave the internal HERWIG soft underlying event turned
off, if you want to get a more realistic underlying event.

The HERWIG parameters are set from the job options service. 
The following is needed if you wish to run HERWIG
\begin{verbatim}
theApp.DLLs  += [ "Herwig_i" ]  
theApp.TopAlg = ["Herwig"]
\end{verbatim}
The parameters are passed via the following line in the jobOptions.py
file.
\begin{verbatim}
Herwig.HerwigCommand = ["variable index value"]
\end{verbatim} 
Here {\bf variable} is a valid HERWIG variable. Consult the HERWIG
documentation for the variable names and what they do. If the {\bf
  variable} is an array, {\bf index} specifies the entry to be
changed. {\bf value} is the value of the variable.  For example
\begin{verbatim}
Herwig.HerwigCommand = ["iproc 1499","modpdf 10042", "autpdf HWLHAPDF"]
\end{verbatim} 
produces $W$ bosons using the CTEQ structure functions. {\bf note that
all variable names are lower case, parameter values are case sensitive
(CTEQ in this example).}
\begin{verbatim}
Herwig.HerwigCommand = ["iproc 2150","modpdf 10042", "autpdf HWLHAPD
", "ptmin 100.", "ptmax 200.", "emmin 80.", "emmax 100."]   
AtRndmGenSvc.Seeds = ["HERWIG 390020611 821000366", "HERWIG_INIT
820021 2347532"]
\end{verbatim}
produces $Z+jet$ events and resets the random number seeds. Note that
the random number seeds are set by the Athena random number service.
All variables that are marked as settable by users in the HERWIG
Manual can be changed. Please report any bugs.

\medskip

\subsection*{Note on using JIMMY for the underyling event}

\noindent
JIMMY is developed as a plug-and-play add-on of HERWIG. It is turned
on by setting the variable {\tt msflag = 1}. The following, JIMMY
specific, variables may also be set: jmbug, ptjim, and jmueo which are
documented on the external JIMMY package documentation page:

{\tt http://projects.hepforge.org/jimmy/ }.

\subsection*{Note regarding SUSY}

\noindent
For simulation of a SUSY process, ISAWIG must first be run to generate
an intput file. ISAWIG is a standalone program \cite{isawig}.
The name of this file is then passed with

\begin{verbatim}
Herwig.HerwigCommand=["susyfile myfile"] 
\end{verbatim}
where myfile is the name of your input file.

\subsection*{Interface to TAUOLA and PHOTOS}

\noindent
HERWIG is interfaced to TAUOLA and PHOTOS in a different way than the
other Generator Modules (Pythia, etc).  To activate TAUOLA and PHOTOS
in HERWIG you simply need to add in you jobOptions file
\begin{verbatim}
Herwig.HerwigCommand += ["taudec TAUOLA"]
\end{verbatim}
For more information on how HERWIG works with TAUOLA, please refer to
the HERWIG manual.

\subsection*{Random Numbers}

\noindent
Herwig\_i/Herwig.cxx is using the AtRndmGenSvc Athena Service to
provide to HERWIG (via the hwrgen function, found in
Herwig\_i/src/hwrgen.F) the necessary random numbers. This service is
using the RanecuEngine of CLHEP, and is based on the ``stream'' logic,
each stream being able to provide an idependent sequence of random
numbers. Herwig.cxx is using two streams: HERWIG\_INIT and HERWIG. The
first stream is used to provide random numbers for the initialization
phase of HERWIG and the second one for the event generation. The user
can set the initial seeds of each stream via the following option in
the jobOption file.

\begin{verbatim} 
AtRndmGenSvc.Seeds = [``HERWIG_INIT 2345533 9922199'', ``HERWIG 5498921 659091'']
\end{verbatim}



The above sets the seeds of the HERWIG\_INIT stream to 2345533 and
9922199 and of the HERWIG one to 5498921 and 659091. If the user will
not set the seeds of a stream then the AtRndmGenSvc will use default
values.

The seeds of the Random number service are saved for each event in
the HepMC Event record and they are printed on screen by DumpMC. In
this way an event can be reproduced easily. The user has to rerun the
job by simply seting the seeds of the HERWIG stream (the seeds of the
HERWIG\_INIT stream should stay the same) to the seeds of that event.

Additionaly the AtRndmGenSvc is dumping into a file (AtRndmGenSvc.out)
the seeds of all the streams at the end of the job. This file can be
read back by the service if the user set the option
{\tt AtRndmGenSvc.ReadFromFile = true;} 
(default = false). In this case the file
AtRndmGenSvc.out is read and the streams saved in this file are
created with seeds as in this file. The name of the file to be read
can be set by the user via the option
{\tt AtRndmGenSvc.FileToRead = MyFileName;}
The above file should also be written out when a job crashes. 
The above options are contained in 
{\tt Herwig\_i/share/jobOptions.herwig.py}.

\subsection*{Note on hadronizing parton level generated events from Les Houches Type generators}

This is supported by Herwig, as explained in Chapter 9.1 of the 6.5 Herwig
manual). External procesess usually read a file containting events.
At present several externals are available, AcerMC, Alpgen, McAtNlo,
MadCUP, MadGraph, GR@PPA, and Charybdis. To find out how to run these
external processes please refer to the documentation of these generators.

\subsection*{Note on LHAPDF structure functions}

\noindent
In the case you want to run HERWIG with the LHAPDF structure functions
you need to set the autpdf variable to HWLHAPDF and the modpdf one to
the LHAPDF set/member index (see the documentation of the
Generators/Lhapdf\_i package for the LHAPDF set/member index
settings).

\begin{thebibliography}{99}
  \bibitem{isawig}http://www-thphys.physics.ox.ac.uk/users/PeterRichardson/HERWIG/isawig.html
\end{thebibliography}

\end{document}





