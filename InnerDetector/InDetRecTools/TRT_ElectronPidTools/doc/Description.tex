\documentclass[a4paper,11pt]{article} 

\author{Simon Heisterkamp} 
\title{PID tool description}
\date{\today}
\begin{document} 

\maketitle 

\tableofcontents 

\begin{abstract}
  This document is meant as a comprehensive introduction to anyone who
  wishes a deepers understanding of the workings of the PID tool. It
  is intended for future and current developers and for those that
  need to keep the calibration up to date. This document is under
  delevopement at the time of the last software tag. To find the
  latest version please look at the svn-trunk.
\end{abstract}

\section{Purpose}
The PID tool has one primary purpose which is to tag tracks as
electrons based mostly on the Transition Radiation (TR) emission that
electrons perform in the TRT. A minor, secondary function is to give
the expected probability of a given hit being a high threshold (HT)
hit this can be used by the Atlfast inner detector simulation.

In the current impementation the tool uses two ways to identify
electrons. The first method works by looking at the HT bit on each hit
and multiplying up the likelyhoods for the obserrvation for both an
electron and a non-electron hypothesis. The return value is the
realtive likelyhood of the electron hypothesis. The second method
looks only at hits that are not HT hits (aka LT hits) and works with
the time over threshold (ToT). The observed ToT is corrected for
signal propagation along the straw and compared with an expectation
based on the length of the track in the straw. Finally the value $\sum
ToT / \sum L$ is compared to an expected distribution for electrons
and non-electrons to get the relative likelyhood of the electron
hypothesis.

\section{Structure} 

\subsection{Overview}
The PID tool is organized in the following way: The PID tool is an
Athena tool woth proper initialization and finaization methods. Upon
creation is instantiates two daughter tools, the HTcaluclator and the
ToTcalculator. These two tools will only exist as a single instance
each, and are meant to encapsulate the calculation and calibration
part as much as possible. They both inherit from a BaseCalculator
which inplements their common features.

After instantiation comes the initialize() phase. Here the PID tool
checks that two objects of the correct types exist in the database,
one containing calibration constants for the HT part, the other for
the ToT part. Currently both are implemented as Coral Blob64k. If the
objects are not found, initialization is declared a failure.

Next the update(\ldots) method is called which retreives those two
blobs, and after an internal structure check (implemented as a layout
version number) the blobs are passed on to the HT- and ToT-caluclators
for them to read their calibration constants out of them. This
concludes the initialization.

When events are being analyzed, each complete track is passed to the
PID tool and the method electronProbability(Trk::Track\&) is
called. The vector of track hits is retreived and iterated over and HT
and ToT information is collected for each hit.

\paragraph{HT} A likelihood value is intialized to $1.0$ for each of
the two identity hypothesis electron and pion. For each hit the
probability is found for the hit being a High Treshold hit for each of
the two hypothesis. If the hit was HT, we multiply the product by this
probability, if the hit was not HT we mutiply by $(1-p)$.
Finally the electorn PID for HT is found as
\begin{equation}
PID_{HT}=\frac{likel_{ele}}{likel_{ele}+likel_{pion}} .
\end{equation}

\paragraph{ToT} We initialize a sum over ToT and a sum over L to $0.0$
and add for each hit the corrected ToT and the expected L. Afterwards,
if the sum over L is non-zero we divide the sum over ToT by the sum
over L and this value $\sum ToT/\sum L$ is our discriminator.  Finally
the ToT calculator contains the known normalized distributions for
this discriminator that we expect for the case of electron and
non-electrons. By comparing with these distributuons we can obtain the
realtive likelyhood of the electron hypothesis $PID_{ToT}$.

\paragraph{Brem} The previous implementation also contained a
likelyhood calculation based on the convergance of Brem-fitters versus
normal fitters. This part is currently omitted since it has not been
validated or checked in the past two years.

\subsection{Return value}

\emph{This has its own section to make it easier to find. }

The return value of the electronProbability(\ldots) function is a stl
vector of exactly four floats. The values are $PID_{comb}$,
$PID_{HT}$, $\sum ToT / \sum L$ and $PID_{Brem}$
repectively. $PID_{Brem}$ is currently fixed to be $0.5$ representing
no information because this part is currently not implemented. $\sum
ToT / \sum L$ is the discriminator of the ToT calculation. It is
desctibed in detail in section \ref{ToTcalculator}.  $PID_{comb}$ is
calculated from the other PID values as follows
\begin{equation}
  \label{PIDsum}
  PID_{comb}=\frac{\displaystyle{\prod_iPID_i}}{\displaystyle{ \prod_iPID_i + \prod_i(1-PID_i)}} .
\end{equation}
It was decided to return the discriminator of the ToT method instead
of the the PID value because we expected the user to either trust all
our best efforts and use the combined best value, or trust only the
simplest part of our calculations and use the HT part only. The
discriminator itself is non-trivial to calculate and may be helpful to
some studies. If the $PID_{ToT}$ value is explicitly required, it can
be found from the others by inverting equation \ref{PIDsum}.

\section{BaseCalculator}
The BaseCalculator is an abstract class which does two things. It
handles the setting and of calibration constants, and it has the
Limit(\ldots) function to limit the returned PID value.

\subsection{Calibration retreival}
Calibration constants are updated in the follwoing way: The PID tool
has an update method which ckecks the detStore for the presence of an
AthenaAttributeList which contains a single blob of bytes. The Blob is
retreived. The first four bytes in the Blob are a version number and a
day, month and year respectively. The version number is used as a
check that the Blob has the correct internal format. When the
constants are updated, this version should not be changed. It should
only be changed when the the internal addressing changes. The day,
month and year information are meant as an information for the user to
see when the calibration was last updated. If the version was found to
be correct, the Blob is passed to the FillBlob(\ldots) method and is
copied byte for byte to an internal array. If a version mismatch is
detected, a bunch of errors are reported and instead of using a
potentially corrupted blob, the tool loads a hard-coded set of
calibration constants. Finally the entire datablob representing the
hard-coded constants is printed to the screen. If this is ever
encountered during normal running it means that the databse contains
outdated, wrong or corrupted data. This should not be ignored,
although it will not cause the PID tool to fail.

\subsection{Setting the Calibration constants}
The hard-coded calibration constants serve a second purpose. When new
calibration constatnts have been determined and one wishes to upload
them to the database it is recommended to use the following method:
\begin{enumerate}
\item Check out the PID tool to run it locally
\item Force the calibration to use the hard-coded values instead of
  the ones from the database. Go to the marker ``UPDATE\_MARKER'' in
  TRT\_ElectronPidTool.cxx and follow the instructions there.
\item Go into the .cxx files and enter the new constants in the method
  setDefaultCalibrationConstants()
\item Run the Tool
\item find the error messages that contain the data blob and copy the
  lists ( [\ldots] )
\item paste the list into the file DatabaseTools / WritePyCoolAll.py
\item make sure that the database name and tags are correct and
  updated
\item run ``python WritePyCoolAll.py''
\item update the online database with the resulting sqlite file
  mycool.db
\end{enumerate}

\subsection{Limit(\ldots)}
The function Limit(float) is common to both PID calculations. It is
needed for the following purpose: After each calculator has returned
what it believes to be the likelyhood of the track being an electron,
equation \ref{PIDsum} is used to combine the results. This equation
has the property that if any of the PID values are close to 0 or 1,
they will force the result to also be close to that value, overruling
the information from the other PID values. Closeness of a PID value to
0 or 1 therby represents a kind of certainty about the outcome that
cannot be overruled by the other values. The limit function limits the
output value to lie within a restricted range, say $0.02-0.98$. the
return value is the same as the input value al long as it lies within
the limits. If it lies outside the limits, the limit is returned. In
this way we say that we do not the trust the calculator to be more
certain in it's evaluation than a certain value.

\section{HT calculator}
The interface from the HT calculator to the PID tool consist of a
single method: getProbHT(\ldots). getProbHT returns the probability
that a hit will be a HT hit. This probabiltu depends on a number of
inputs:
\begin{itemize}
  \item the momentum of the Track
  \item the assumed identity of the track (electron or not)
  \item which straw is being hit
  %\item where along the straw the hit 
\end{itemize}

The calculation of this probability is primarily based on the onset
curve of Transition Radiation emission as a function of the
$\beta\gamma$ of the track. The onset curve is parametrized as a the
sum of a linear part and a logistic sigmoid part. The equation is
\begin{eqnarray}{l}
  P_{HT}=\underbrace{[0]+x\times[1]}_{\frac{dE}{dx}\mbox{ part}}+\underbrace{[2]*\left(1+exp\left(\frac{x-[3]}{[4]}\right)\right)^{-1}}_{\mbox{TR part}}\\
  \mbox{where }x=log_{10}(\beta\gamma).
\end{eqnarray}
The $[n]$ objects simply represent parameters that are tuned to a

Currently there is one set of these parameters per detector section:
Barrel, Endcap A-Wheels and Endcap B-Wheels. Between these sections is
where we see the greatest variation of HT prbability. There is however
also variation within the modules, and therefore the HT calculator
contains two additional calibration constants per Strawlayer, one that
is multiplyed with the whole TR part, and one that is multiplied with
the $dE/dx$ part.

\subsection{Theoughts and Discussion}

Therer are some alternative ways in which one might have chosen to
calculate the likelyhood of the electron hypthesis. For example, the
absolute HT chance is not required, only the chance for electrons
relative to pions of the same momentum. The reason why this method was
chosen is to allow its use in the Atlfast inner detector simulation
instead of implementing that in another tool. If it is found that
other methods are clearer, better or faster it might be woth
considering to make the $P_{HT}$ into a separate part or even another
tool.

\section{ToT calculator}
\label{ToTcalculator}

The ToT calculator is somewhat more complicated than the HT
calculator. The likelyhood based on ToT is calculated in the following
way: For each LT hit the raw ToT is extracted from the largest island
of continuous LT bits in the bitpattern. This was found to give the
best separation as opposed to using the leading edge to trailing edge
distance. The ToT is then corrected for signal propagation
effects. The source of these propagation effects is that from the hit
location two electric pulses travel along the straw, one towards the
electronics, the other towards the insulated end where it is reflected
back into the electronics. A hit that occurs closer to the
instrumented end will be detected somewhat sooner with its outgoing
pulse returning later. This gives a longer ToT to hits that occur near
the electronics as opposed to those that occur at the closed
end. Based on this idea three equations were devised to correct the
ToT to a value that has a uniform distribution everywhere in the
detector. The equations are
\begin{equation}
\label{shortstrawcorr}
  - [0] +\frac{z^2 -697.1}{[1]} * exp( \frac{z^2-704.6}{[2]})
\end{equation}
for the short barrel straws,
\begin{equation}
  -[0] + \frac{|z|-697.1}{[1]} * exp(\frac{|z|-704.6}{[2]})
\end{equation}
for the long barrel straws, and
\begin{equation}
  \label{endcapcorr}
  -[0] +[1]*exp(\frac{r^2-[2]}{[3]})
\end{equation}
for the endcap.  These equations are used to calculate an additive
correction to the ToT. Notice that this value is independent of the
actual ToT value.

\end{document}