% $Id: CoralDB-Design.tex,v 1.9 2008-03-03 21:39:27 beringer Exp $

\begingroup
%\newcommand{\sql}[1]{\textbf{#1}}
\newcommand{\sql}[1]{\texttt{#1}}

\subsection{Design}

% Main author: Juerg

When thinking about modelling connectivity, two fundamentally
different approaches come to mind:

On one hand, one could model the configuration data for each type of
object such as e.g.\ modules in a specific database table and use
foreign keys (i.e.\ references to entries in other database tables) in
order to provide information about connections to other objects. For
example, the fact that a module is connected to a high-voltage power
supply could be modelled by a column such as HV\_POWER\_SUPPLY in the module
table, whose entries would then be references to entries in a
high-voltage power supply table. In this scheme the physical
connections are reflected explicitly in the database schema, and by
using constraints the database can be used to ensure the integrity of
the connectivity data.  Thus mistakes such as connecting a module to
two high-voltage power supplies instead of a high-voltage and a
low-voltage power supply can be prevented at the database level.

On the other hand, one might choose a generic approach where
connectivity data is stored as a list of connections between objects
using a single database table. In such a table all connections are
treated identically, and no protection can be provided at the level of
the database schema against entering connections that do not make any sense. The
physical connections are not reflected in the database schema.
Therefore no schema change is necessary if the connectivity model
changes.

Both approaches have their advantages. The explicit modelling of the
connectivity in the first case is very efficient with a relational
database, queries involving different connections can easily be
optimized, and database constraints can be fully exploited to ensure
the consistency and completeness of the connectivity information. The
generic approach is much more flexible, changes in the physical or
logical connectivity do no require database schema changes, but unless one
wants to exploit recursive queries which are not supported in a
standard way by relational databases, traversing each connection
between two objects requires a distinct database query.

One of the specific requirements for the pixel connectivity database
was support of a generic object-oriented interface called
PixDbInterface (see section \ref{PixDbInterface}) for looking up connectivity information. The generality
of this interface does not allow an explicit modelling of the
connectivity information in the database schema, and therefore the
generic approach described above was chosen for the pixel connectivity
database. This led to the development of CoralDB, and to the implementation
of PixDbInterface with CoralDB (called PixDbCoralDB).

In CoralDB objects are represented by their ID, which is simply a string.
Different objects have different IDs. Each object is assigned a type name
(another string) and is associated with
\begin{itemize}
\item an optional set of connections from this object to other objects,
\item an optional set of connections from other objects to this object,
\item an optional set of alias names,
\item and optional payload data consisting of an arbitrary number
of named payload fields.
\end{itemize}

A connection between two objects is represented by the
the ID of the object where the connection originates, by the ID of the object
where the connection terminates (thus all connections have a direction), and by
a ``slot'' name at both ends of the connection. The slot name is a string that
is used as a generic index in order to distinguish different connections originating
or terminating at a given object, and to store information such as channels, slot
number in a crate, etc.\ The combination of ID and slot name is unique for outgoing
connections, and the same is true for incoming connections. Thus an incoming or outgoing
connection is identified by the pair (ID,slot), and a connection is uniquely identified
by the tuple (fromID,fromSlot,toID,toSlot).

Alias names are assigned to different categories called ``conventions''. Within each convention,
each object can have at most one alias. The number of conventions is not limited.

In order to avoid trivial mistakes such as mistyping an object name, a catalog
of all objects and their types is kept (``object dictionary'').
The database ensures that connections, aliases and payload fields
can only reference objects in this catalog. The
object dictionary may contain objects that are not referenced by the connectivity,
alias or payload tables, allowing to add spare objects (which may be physically present but not
currently connected and used) to a connectivity database.

CoralDB supports different versions of connectivity within the same database schema (Oracle)
or database file (SQLite). Several versions of the object dictionary can coexist without any naming
conflicts between different versions. For each version of the object dictionary, several versions
of connectivity, aliases and payload data may exist that all share the same set of objects. These
versions are independend from each other apart from using the same object definition. Thus when making
a change to a payload data field, no change in the connectivity or alias version is required.

In this note the object dictionary tag is sometimes called IDTAG, to
distinguish it from the other tags.  The contents of different IDTAGs
is completely independent.  For example, IDTAG1 and IDTAG2 can have
connectivity tags with the same name, e.g. CTAG.  CTAG in IDTAG1 is
not related in any way to CTAG in IDTAG2.


\subsection{Database Schema}

% Main author: Andrei

\NFIG{tb}
%{schemav2.eps}{width=\textwidth}
{schemav2-oracle-replica.eps}{width=\textwidth}
{schema}{CoralDB database schema}
{Database schema used by CoralDB. The solid lines between tables represent foreign key contraints.}


A graphical representation of the database schema is shown on
Fig.~\ref{schema}.  Database tables are represented as rectangles,
with the table name in the top part, and column names and types in the
bottom part of each rectangle.  The arrows on the picture represent
database foreign key constraints.  The names of all tables for the
current database version start with the \sql{CORALDB2\_} prefix.

The object dictionary table, \sql{CORALDB2\_OBJECTDICT}, contains a list
of all objects (the \sql{ID} column) and their types (\sql{TYPE}).
The support for different versions (``tags'') of object dictionary is
implemented via the additional column \sql{IDTAG} to record an object
dictionary tag.  An object is uniquely identified by the pair of
strings (\sql{IDTAG}, \sql{ID}), and this pair is the table's primary
key.  Foreign key constraints in several other CoralDB tables ensure
completeness of the object dictionary, by requiring that any occurence
of \sql{ID} and its \sql{IDTAG} reference the corresponding columns in
\sql{CORALDB2\_OBJECTDICT}.

Connections are stored in the \sql{CORALDB2\_CONNECTIONS} table.  It
contains the \sql{ID},\sql{SLOT} and \sql{TOID},\sql{TOSLOT} columns
to represent a connection, the \sql{IDTAG} column to specify an object
dictionary tag (for both \sql{ID} and \sql{TOID} --- there are two
foreign keys to \sql{CORALDB2\_OBJECTDICT}), and a \sql{TAG} column to
specify a connectivity tag.  A connection can be uniquely identified
by either (\sql{ID},\sql{IDTAG},\sql{TAG},\sql{SLOT}), or by
(\sql{TOID},\sql{IDTAG},\sql{TAG},\sql{TOSLOT}) combinations.

The \sql{CORALDB2\_MASTERLIST} keeps a subset of objects per
connectivity tag.  These objects are intended to be used as a starting
point to navigate the connectivity structure.  PixDbInterface uses
this table to store a ``root record''.  For \sql{CORALDB2\_MASTERLIST}
to be usable with PixDbInterface, it must contain a single entry for
the connectivity tag in question.  (Otherwise to access the
connectivity tag, one would need to explicitly specify a ``root
record'' when initializing PixDbInterface.)

The \sql{CORALDB2\_ALIASES} table keeps the aliases.  Different
versions of aliases for a given ID are distinguished by the \sql{TAG}
column, that records the alias tag.

For the storage of payload data, a slightly more elaborate versioning
schema is implemented.  The motivation is to make possible sharing of
payload data among different data tags.  This possibility is not used
by the existing code, but it can be implemented with only C++ code
modifications, without any database schema changes.  The table
\sql{CORALDB2\_CLOBS\_REVS} establishes a correspondence between a
data tag (\sql{TAG}), and a numeric revision (\sql{REV}) of data for
an \sql{ID}.  The actual payload data for a field \sql{FIELD} of
object \sql{ID} are stored in the \sql{DATA} column of the
\sql{CORALDB2\_CLOBS} table.  To identify a specific version of
payload data, the revision number \sql{REV} is used that corresponds
to \sql{REV} in the \sql{CORALDB2\_CLOBS\_REVS} table.  Note that the
correspondence between rows of \sql{CORALDB2\_CLOBS} and
\sql{CORALDB2\_CLOBS\_REVS} tables are many-to-many; therefore it is
not possible to have a foreing key in any directions.  The actual
values of revision numbers are not accessible through the CoralDB
interface, and their use is completely transparent to client code.

The tables \sql{CORALDB2\_CONNECTIONS\_TAGS},
\sql{CORALDB2\_ALIASES\_TAGS}, \sql{CORALDB2\_DATA\_TAGS}, contain
complete (by foreign keys) lists of corresponding tags.  The primary
key in each table is the (\sql{IDTAG},\sql{TAG}) combination, so there
are no constraints on tag names in different object dictionary tags.
The \sql{LOCKED} boolean is set to \texttt{false} when a new tag is
created.  If it is set to \texttt{true}, the C++ code will refuse to
make any modifications to the database records marked with this tag.

The \sql{CORALDB2\_OBJECTDICT\_TAGS} table contains a complete list of
object dictionary tags (\sql{IDTAG}), along with their write
protection status (\sql{LOCKED}).  CoralDB code does not check whether
an IDTAG is unlocked for each access, it only checks whether the
appropriate connectivity/data/alias tag is unlocked.  However a
connectivity/data/alias tag can be locked only if the corresponding
object dictionary tag is locked.

\textbf{Note that there are no safeguards to prevent modifications of
  ``locked'' data at the database level.  Locking therefore is merely
  advisory, and is only enforced by the C++ access layer.}

At the time of this writing, all strings in CoralDB such as object names, alias names, tag
names etc.\ are limited to 100 characters. The length of each type of string is
defined by a constant in the CoralDB header file (e.g.\  {\tt MAX\_ID\_LENGTH}), so
the lenghts of different types of strings could be optimized individually in the
future if needed.

%Length constraints on different strings in CoralDB are:
%\begin{verbatim}
%   static const int MAX_ID_LENGTH = 50;
%   static const int MAX_ALIAS_LENGTH = 50;
%   static const int MAX_CONVENTION_LENGTH = 50;
%   static const int MAX_SLOT_LENGTH = 40;
%   static const int MAX_FIELDNAME_LENGTH = 40;
%   static const int MAX_TAG_LENGTH = 60;
%   static const int MAX_OBJECTTYPE_LENGTH = 40;
%\end{verbatim}

\endgroup
